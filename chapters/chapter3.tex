\chapter{Integer domains}
In this chapter we focus on the domain of integers, and try to determine sufficient conditions for the non existence of under-approximation abstract domains.

\section{Sum completeness}
The concept of complete abstraction (Definition \ref{ch2:def:complete-abstr}) for a function is very important in over-approximation abstract interpretation because it correspond to analyses without false alarms. In general it's very hard that all the basic transfer functions of a program are complete, and in particular boolean guards are almost never so. However, it's not uncommon that \textit{some} of them are complete: for instance, in the interval domain sum is a complete operation.

In under-approximation, completeness is much harder to achieve: the reason is the fact that complete abstractions allow to ``distribute" $\alpha$ over the operation, so having even a single set that is abstracted to $\bot$ forces, by strictness of concrete operations, all concrete sets that can be obtained applying the operation to that specific set to be $\bot$ as well. This idea is more easily seen in the proof of the following proposition.
\begin{prop}\label{ch3:th:sum-complete-trivial}
	Let $\ugi{\pow(\setZ)}{\alpha}{A}$ be an under-approximating Galois insertion. If $A$ is complete for the concrete sum then it's trivial.
\end{prop}
In this proposition we're considering as usual the additive extension of $+$. An under-approximation abstract domain $A$ is said to be trivial if it is the concrete domain or it only contains $\bot$.
\begin{proof}
	Let $+^{\flat}$ be the complete abstraction of $+$.
	Let us distinguish two cases.

	If all singletons $\{ n \}$ are representable in $A$ then, by union closure (Proposition \ref{ch2:th:under-gc-union-closure}) we get that any $S \in \pow(\setZ)$ is also in $A$ since it's union of singletons. This in turn entails $A = \pow(\setZ)$, that is trivial.

	Otherwise, there exists an integer $\bar{n}$ such that $\alpha(\{ \bar{n} \}) = \emptyset$. Consider then an arbitrary set of integers $S \subseteq \setZ$.
	\begin{align*}
		\alpha(S) &= \alpha(S - \{ \bar{n} \} + \{ \bar{n} \} ) \\
		&= \alpha(S - \{ \bar{n} \}) +^{\flat} \alpha(\{ \bar{n} \}) \\
		&= \alpha(S - \{ \bar{n} \}) +^{\flat} \emptyset \\
		&\subseteq \alpha(S - \{ \bar{n} \}) + \emptyset = \emptyset
	\end{align*}
	where the second line follows by completeness of $+^{\flat}$ and the last by its correctness. This means that any set $S$ is abstracted to $\emptyset$, so $A = \{ \emptyset \}$.
\end{proof}

In our opinion, the reason this doesn't happen in over-approximation is an instance of the asymmetry born of basic transfer functions. As we introduced in the previous chapter, many operations, defined as the additive extension of basic constructs, increase the output cardinality. For instance, the sum of two sets of integers have a cardinality that is at least the largest of the two inputs. This asymmetry allows to get many subsets of the concrete domain as the result of applying the function to a small set (for instance a singleton) but not to a big one.
Then in under-approximation small sets can be composed by union to make the abstract domain grow until it's too big to be feasible, while in over-approximation sets are composed by intersection. To emulate this behaviour, we would need big sets (eg. the whole $C$ without a single value) so that their intersections give rise to lots of smaller sets, but this conflicts with the ``direction" of basic transfer functions.

Of course in general completeness is a very strong property to require, so in the next section we'll introduce a weaker notion and show that it is enough to prevent the existence of under-approximation abstract domains.

\section{Non emptyingness}
Completeness is a concept tailored to over-approximation: we use it to avoid false alarms while over-approximating the set of possible values. In bug catching, an alternative, weaker definition may be enough
\begin{definition}[Non emptying]\label{ch3:def:non-emptying}
	Let $\ugc{C}{\alpha}{\gamma}{A}$ be an under-approximation Galois connection, $f : C \rightarrow C$ a monotone function on $C$ and $f^{\flat} = \alpha \circ f \circ \gamma$ its best correct approximation in $A$.
	We say that $f$ is \textit{non emptying} (in $A$) if, for any concrete value $c$, if both $\alpha(c) \neq \bot$ and $\alpha(f(c)) \neq \bot$ then also $f^{\flat}(\alpha(c)) \neq \bot$.
\end{definition}

Unlike completeness, this definition doesn't mean that the analysis will find the best possible result the abstraction can, but just that if it starts from something ($\alpha(c) \neq \bot$) and it can find something ($\alpha(f(c)) \neq \bot$) then it will find at least one of the possible results ($f^{\flat}(\alpha(c)) \neq \bot$).
The rationale behind it is that the analysis shouldn't get to $\bot$ because, as stated in the previous chapter, recovery from it is hard. The definition exactly prevents this when it's caused only by imprecision of the abstracted function.

Clearly completeness implies non emptyingness since $f^{\flat}(\alpha(c)) = \alpha(f(c)) \neq \bot$, but the converse is not true, as shown in the following example.
\begin{example}\label{ch3:ex:ne-not-complete}
	Consider the under-approximation domain $\Int_0$ of intervals containing $0$ (Example \ref{ch2:ex:intervals-0}). Do note that in this example we write intervals to denote subsets of $\setZ$, following the identification given by the Galois insertion, as well as use both $\bot$ and $\emptyset$ to refer to the same object (the empty subset of $\setZ$).

	Consider the additive extension of the function $f(x) = x + 1$. This function is not complete in the abstract domain since, for instance on the concrete element $S = \{ -1 \}$, we have
	\begin{align*}
		&\alpha(f(S)) = \alpha(\{ 0 \}) = [0, 0] \\
		&f^{\flat}(\alpha(S)) = f^{\flat}(\bot) = \bot
	\end{align*}

	However this function is non emptying.
	First we show that any concrete element $S \in \pow(\setZ)$ is such that $\alpha(S) \neq \emptyset$ if and only if $0 \in S$.
	Suppose that $0 \in S$. Then by monotonicity of $\alpha$
	\[
	\alpha(S) \supseteq \alpha(\{ 0 \}) = [0, 0] \supset \emptyset
	\]
	Conversely, assume that $\alpha(S) \neq \emptyset$. Since all elements of $\Int_0$ but the empty set contains $0$, and $\alpha \preceq \id$, we get
	\[
	0 \in \alpha(S) \subseteq S
	\]
	that is exactly $0 \in S$.

	Consider now an arbitrary concrete element $S$ such that $\alpha(S) = \bot$ and $\alpha(f(S)) = \bot$. As shown above, the former condition is equivalent to $0 \in S$, and the latter to $0 \in f(S)$, that is in turn equivalent to $-1 \in S$. Using these two hypothesis, we can show
	\begin{align*}
		&S \supseteq \{ -1, 0 \} \\
		\implies& \alpha(S) \supseteq \alpha(\{ -1, 0 \}) = [-1, 0] \\
		\implies& f(\alpha(S)) \supseteq f([-1, 0]) = [0, 1] \\
		\implies& f^{\flat}(\alpha(S)) = \alpha(f(\alpha(S))) \supseteq \alpha([0, 1]) = [0, 1] \supset \emptyset
	\end{align*}
	that is exactly $f^{\flat}(\alpha(S)) \neq \bot$, the non emptying condition.
\end{example}
However, even this weaker notion allows to prove some results on $A$ under some assumptions, as we show in the remainder of this chapter.

We assume there is an under-approximation Galois insertion $\ugi{\pow(C)}{\alpha}{A}$. Moreover, we say an element $S \in \pow(C)$ is \textit{representable} if it belongs to $A$, or equivalently if $\alpha(S) = S$.

\begin{definition}\label{ch3:def:repr-with-set}
	Let $S \subseteq C$ be a subset of $C$. We say that $d \in C$ is \textit{representable with $S$} if $S \cup \{ d \}$ is representable. We call $R(S)$ the set of elements of $C$ representable with $S$, ie.
	\[
	R(S) = \{ d \in C \svert \alpha(\{ d \} \cup S) = \{ d \} \cup S \}
	\]
\end{definition}
For the sake of brevity, we shall write $R$ for $R(\emptyset)$, the set of representable values of $C$, and $R(c)$ for $R(\{ c \})$ where $c \in C$ is any concrete value.

We now present a lemma about non emptying functions that makes reasoning easier. This lemma is weaker than the definition, but is nevertheless one of the main tools we use when considering non emptying functions.

\begin{lemma}\label{ch3:th:f-non-repr-pair}
	Let $f: C \rightarrow C$ be non emptying, $c \in R$ and the pair $\{ c, \bar{c} \}$ be not representable, ie. $\bar{c} \notin R(c)$. If $f(\bar{c}) \in R$ then also $f(c) \in R$.
\end{lemma}
\begin{proof}
	Since $\{ c, \bar{c} \} \supseteq \{ c \}$ we have
	\[
	\alpha(\{ c, \bar{c} \}) \supseteq \alpha(\{ c \}) = \{ c \}
	\]
	where the equality follows because $c \in R$ is representable. Since by correctness
	\[
	\{ c, \bar{c} \} \supseteq \alpha(\{ c, \bar{c} \})
	\]
	and $\alpha(\{ c, \bar{c} \})$ can't be the pair $\{ c, \bar{c} \}$ because this is not representable and hence not in the image of $\alpha$, it should be the case that $\alpha(\{ c, \bar{c} \}) = \{ c \}$.

	Now
	\[
	\alpha(f(\{ c, \bar{c} \})) = \alpha(\{ f(c), f(\bar{c}) \}) \supseteq \alpha(\{ f(\bar{c}) \}) = \{ f(\bar{c}) \}
	\]
	where the last equality follows by the hypothesis that $f(\bar{c}) \in R$.
	This in particular means that $\alpha(f(\{ c, \bar{c} \})) \neq \emptyset$, and together with the fact that $\alpha(\{ c, \bar{c} \}) = \{ c \} \neq \emptyset$, since $f$ is non emptying we get that
	\[
	f^{\flat}(\alpha(\{ c, \bar{c} \})) \neq \emptyset
	\]

	From this we find
	\begin{align*}
		\emptyset &\subset f^{\flat}(\alpha(\{ c, \bar{c} \})) \\
		&= f^{\flat}(\{ c \}) \\
		&= \alpha \circ f(\{ c \}) \\
		&= \alpha(\{ f(c) \})
	\end{align*}
	Again by correctness we have $\alpha(\{ f(c) \}) \subseteq \{ f(c)\}$, and since this can't be empty it should be exactly $\alpha(\{ f(c) \}) = \{ f(c) \}$, that is $f(c) \in R$.
\end{proof}

The goal of this lemma is to use non emptying functions to get new representable elements, as this will be our main tool to prove non existence of under-approximation abstract domains.

We now apply this definition on integer domains in two different situations, on the infinite domain $\pow(\setZ)$ and on the finite $\pow([-N; N])$ of machine integers.

\subsection{Infinite domain}
In this subsection we focus on the infinite concrete domain $\pow(\setZ)$. We assume that an abstract domain $A$, to be feasible for analyses, must be at most countable, that is the same size as $\setZ$. We do this assumption because we want to represent abstract elements with an amount of bits comparable with that of concrete \textit{values}, to have a complexity comparable with a single concrete execution of the program and not exponentially greater.

We first present a simple cardinality estimate that will be useful in the proof of the following result. The goal of this lemma is to show that some sets must be ``small" (in some sense, in this case finite), so we can find a contradiction showing that one of these sets is actually ``big". We follow this line of reasoning in almost all of our proofs of non existence of abstract domain with some properties.

\begin{lemma}\label{ch3:th:R-S-bound-integer-inf}
	For any fixed subset $S \subseteq \setZ$, $R(S)$ is finite.
\end{lemma}
\begin{proof}
	By union closure of the abstract domain (Proposition \ref{ch2:th:under-gc-union-closure}), any set $S \cup T$ for $T \subseteq R(S)$ is representable too, since it can be expressed as the union of representable sets:
	\[
	S \cup T = \bigcup\limits_{x \in T} (S \cup \{ x \})
	\]
	and $S \cup \{ x \}$ is representable since $x \in R(s)$.

	The number of those sets is the cardinality of $\pow(R(S))$, and since $A$ is at most countable the set $R(S)$ can't be infinite.
\end{proof}

We now present the first result that shows that requiring some functions to be non emptying makes impossible to create an abstract domain. The main idea is to define an infinite sequence of representable elements, that is in contradiction with the previous lemma that says that $R = R(\emptyset)$ is finite.
In order to define such a sequence, we want to use Lemma \ref{ch3:th:f-non-repr-pair}: we start from an initial representable $n_0$ and from a value $\bar{n}$ not representable with it, then find a non-emptying $f$ that maps $\bar{n}$ into $n_0$, so that $f(\bar{n})$ is representable and we can then apply the lemma to get the new representable element $f(n_0)$. We then iterate this procedure, changing $f$, to build the infinite sequence.

We believe the assumption that there exists an initial representable value is not very restrictive since initializations like \code{x = 0} must be abstracted to $\bot$ if $0$ is not representable.

\begin{prop}\label{ch3:th:ne-sum-nonexsistence-inf}
	Let $\ugi{\pow(\setZ)}{\alpha}{A}$ be an under-approximation Galois insertion, and assume that there is an integer $n_0$ that is representable. Then it can't be the case that all the functions of the form $f_n(x) = x + n$ are non emptying in $A$.
\end{prop}

\begin{proof}
	Assume by contradiction that all $f_n$ are non emptying in $A$.
	By hypothesis, $n_0 \in R$, and $R(n_0)$ is at most finite by Lemma \ref{ch3:th:R-S-bound-integer-inf}. Since $\setZ$ is infinite, this means there exists an $\bar{n} \in \setZ \setminus R(n_0)$, that is an element such that the pair $\{ n_0, \bar{n} \}$ is not representable.

	Let $d = n_0 - \bar{n}$ and consider $f_d$. We assumed it to be non emptying, so we can apply Lemma \ref{ch3:th:f-non-repr-pair}: $n_0$ is representable while the pair $\{ n_0, \bar{n} \}$ is not, and
	\[
	f_d(\bar{n}) = \bar{n} + d = \bar{n} + n_0 - \bar{n} = n_0
	\]
	so it's representable. Hence also $f_d(n_0) = n_0 + d$ is representable.

	Following this idea, we can prove by induction on $t$ that $n_0 + t d$ is representable for all $t$. The base step $t = 0$ is the hypothesis that $n_0$ is representable.
	For the inductive step, assume $n_0 + (t - 1) d$ is representable, and consider $f_{t d}$. We assumed this non emptying, so we can apply again Lemma \ref{ch3:th:f-non-repr-pair} to the pair $\{ n_0, \bar{n} \}$:
	\[
	f_{t d}(\bar{n}) = \bar{n} + t d = \bar{n} + n_0 - \bar{n} + (t - 1) d = n_0 + (t - 1) d
	\]
	that is representable by inductive hypothesis. So we get that $f_{t d}(n_0) = n_0 + t d$ is representable too, that is exactly the inductive step.

	Since $\bar{n}$ is not representable while $n_0$ is, we have $n_0 \neq \bar{n}$, or equivalently $d \neq 0$, hence $\{ n_0 + t d \svert t \in \setN \}$ is infinite.
	Moreover $\{ n_0 + t d \svert t \in \setN \} \subseteq R$ by the induction above, but this is impossible since $R$ must be finite by Lemma \ref{ch3:th:R-S-bound-integer-inf}.
\end{proof}

\subsection{Finite domain}
Now we move to a slightly different setting: we assume as concrete domain $\pow([-N, N])$, the power set of a finite, symmetric interval for some large integer value $N$, and we assume all operations are performed ``in machine arithmetic", that is whenever the result is greater than $N$ it wraps back to $-N$ because of overflows. This correspond to work modulo $2N + 1$ taking the unique representative of each congruence class in the interval $[-N, N]$ of interest.
Again, we make an assumption on the size of the abstract domain for the analysis to be feasible: we require the size of $A$ to be polynomial in $N$. This means the number of bits required to represent an abstract element is linear in that for concrete elements.

In this section we'll use asymptotic notation for some quantities. For this to be completely formal we should define a sequence of abstract domain $A_N$, each one for the concrete domain $\pow([-N, N])$, then define a sequence of values for each quantity we want to estimate, and take the limit of this sequence for $N$ going to infinity. However we do believe all these formal details would clutter both statements and proofs, making hard to get insight. For this reason, we avoid all this, just (ab)using the intuitive meaning associated with the notation.

\begin{lemma}\label{ch3:th:R-S-bound-integer-fin}
	For any fixed subset $S \subseteq \setZ$, $\lvert R(S) \rvert = O(\log(N))$.
\end{lemma}
\begin{proof}
	As in the proof of Lemma \ref{ch3:th:R-S-bound-integer-inf}, by union closure any set $S \cup T$ for $T \subseteq R(S)$ is representable. We then have
	\[
	\poly(N) = \abs{A} \ge \abs{\pow(R(S))} = 2^{\abs{R(S)}}
	\]
	so, taking log at both sides, $\abs{R(S))} = O(\log(N))$.
\end{proof}

The following proposition uses the same proof line as Proposition \ref{ch3:th:ne-sum-nonexsistence-inf} above: we define a sequence of representable elements, and prove that they are too much since, by the previous lemma, $R$ is quite small.

\begin{prop}\label{ch3:th:ne-sum-nonexsistence-fin}
	Let $\ugi{\pow([-N, N])}{\alpha}{A}$ be an under-approximation Galois insertion, and assume that there is an integer $n_0$ that is representable. Then it can't be the case that all the functions of the form $f_n(x) = x + n$ (modulo $2N + 1$) are non emptying in $A$.
\end{prop}
\begin{proof}
	Let $r = \abs{R(n_0)}$. By the previous lemma \ref{ch3:th:R-S-bound-integer-fin} we know that $r = O(\log(N))$. Fix an element $\bar{n} \notin R(n_0)$ not representable with $n_0$ such that
	\[
	d = n_0 - \bar{n} \le r + 1
	\]
	This element should exists because otherwise all elements in the interval $[n_0 - r - 1; n_0 - 1]$ (modulo $2N + 1$) would be representable with $n_0$, that is impossible since they are $r + 1 = \abs{R(n_0)} + 1$.

	Following the proof of proposition \ref{ch3:th:ne-sum-nonexsistence-inf}, we can show by induction that for all $t \ge 1$ the value $f_{td}(\bar{n}) = n_0 + (t - 1) d$ is representable. However, all these values are different from one another for
	\[
	1 \le t < \frac{2N + 1}{d}
	\]
	and we also know that
	\[
	\frac{2N + 1}{d} > \frac{2N}{r + 1} = \frac{N}{O(\log(N))}
	\]
	But this is a contradiction since all these values are representable while $\abs{R} = O(\log(N))$ by lemma \ref{ch3:th:R-S-bound-integer-fin}, and
	\[
	\frac{N}{\log(N)} = \omega( \log(N) )
	\]
\end{proof}
