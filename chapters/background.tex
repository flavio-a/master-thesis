\chapter{Background}

Things needed to understand the thesis that aren't strictly related to the thesis subject.

\section{Notation}
\todo{probably drop this section}
$C (\alpha, \gamma) A$ means a Galois Connection (GC) between $C$ and $A$.

If $S$ is a poset, $S^{\op}$ is the poset with the opposite ordering. So for instance $\pow(C)^{\op}$ is the powerset of $C$ ordered by containment $\supseteq$, and hence $\emptyset$ is the greatest element ($\top$) of the set.

Unless otherwise specified, a Galois Insertion (GI) is always specified with the ``small" set on the right, ie. if $C (\alpha, \gamma) A$ is a GI then $\alpha \circ \gamma = \id_A$.

Given a GC $C (\alpha, \gamma) A$, an element $c \in C$ is \textit{representable} in $A$ if $\gamma(\alpha(c)) = c$.

%Given $\pZop (\gamma, \alpha) A$ a GI, we call $n^{\flat}$ the only abstract element (if any) such that $\gamma(n^{\flat}) = \{ n \}$. We say $n^{\flat} \notin A$ if there is no such element in $A$.

Recall that in a GI $\gamma$ and $\alpha$ map concrete top in abstract top and vice versa.

A GI is the same as assuming $A \subseteq C$, and we'll do this. With this notation, $f^{\#}$ is the same as $\alpha \circ f$, but is conceptually different.

Given a function $f : C \rightarrow D$ we use the same symbol also to denote its additive extension $f: \pow(C) \rightarrow \pow(D)$.

Meaning of $\infexists$.

\section{Partially ordered sets}
This section recalls basic notions of order theory upon which (much of) the abstract interpretation framework is based. For a more comprehensive introduction to order theory, we refer the reader to a text book, such as for instance \cite{order-theory-book} or appendix A of \cite{principles-of-program-analysis-book}.

\begin{definition}[Partial order]
	Given a set $S$, a partial order $\preceq$ on $S$ is a relation on it that, for all $a$, $b$ and $c$ in $S$, satisfies
	\begin{itemize}
		\labelitem{reflexivity} $a \preceq a$
		\labelitem{antisymmetry} if both $a \preceq b$ and $b \preceq a$ then $a = b$
		\labelitem{transitivity} if both $a \preceq b$ and $b \preceq c$ then also $a \preceq c$
	\end{itemize}
\end{definition}
We say that the pair $(S, \preceq)$ is a partially ordered set, or a \textit{poset} for short, and we usually write only the carrier set $S$ when the ordering is unambiguous.

\begin{definition}[Opposite ordering]
	Given a poset $(S, \preceq)$, the \textit{opposite} poset $(S, \preceq^{-1})$ is defined with $a \preceq^{-1} b$ if and only if $b \preceq a$.
\end{definition}
We often use $\succeq$ to indicate $\preceq^{-1}$ and $S^{\op}$ for (the carrier set of) the opposite poset.

\begin{definition}[Upper bounds and least upper bound]
	Given a poset $S$ and one of its subsets $T \subseteq S$, an \textit{upper bound} of $T$ is an element $c \in S$ that is greater or equal than any element of $T$:
	\[
	\forall a \in T.\ a \preceq c
	\]
	The \textit{least upper bound} (or lub for short) of $T$, if it exists, is an upper bound of $T$ that is smaller or equal than all other upper bounds of $T$.
\end{definition}
In general a set $T$ needs not have a least upper bound, but when it does it's unique and we denote it with $\bigsqcup T$. Moreover, when $T = \{ a, b \}$ is made of just two elements, we shall write $a \sqcup b$ for their least upper bound.

The dual notion of lub is that of glb:
\begin{definition}[Lower bounds and greatest lower bound]
	Given a poset $S$ and one of its subsets $T \subseteq S$, a \textit{lower bound} of $T$ is an element $c \in S$ that is smaller or equal than any element of $T$:
	\[
	\forall a \in T.\ c \preceq a
	\]
	The \textit{greatest lower bound} (or glb for short) of $T$, if it exists, is a lower bound of $T$ that is greater or equal than all other upper bounds of $T$.
\end{definition}
Again, if this exists we denote it with $\bigsqcap T$, and if $T = \{ a, b \}$ we use the notation $a \sqcap b$.

\begin{definition}[Lattice]
	A poset $S$ is called a \textit{lattice} if every pair of elements has a lub and glb. It is called a \textit{complete lattice} if every subset has a lub and a glb.
\end{definition}

\begin{definition}[Monotone function]
	Given two poset $S$, $T$ and a function $f : S \rightarrow T$, that function is called \textit{monotone} (or \textit{order-preserving}) if, for any pair $a, b \in S$ of elements of the domain such that $a \preceq_S b$, also their images satisfies $f(a) \preceq_T f(b)$.
\end{definition}

Given a set $S$ and a poset $T$, we can consider the set of functions from $S$ to $T$. This has a natural structure of poset too.
\begin{definition}
	Given two functions $f, g: S \rightarrow T$, we say that $f \preceq g$ if for all elements $a \in S$ we have
	\[
	f(a) \preceq g(a)
	\]
\end{definition}
It's easy to show that this relation among functions is a partial order too. Moreover, if $T$ is a (complete) lattice, the set of functions from $S$ to $T$ is a (complete) lattice too, and this still holds if $S$ is a poset and we restrict ourselves to monotone functions between the two.

\section{Galois connections}
\todo{add references for this whole section}
The main mathematical tool we use to study abstract interpretations are Galois connections, and the special case of Galois insertions.

\begin{definition}[Galois connection]\label{ch2:def:gc}
	Let $C$ and $A$ be two partially ordered sets, and $\alpha : C \rightarrow A$, $\gamma : A \rightarrow C$ be a pair of monotone functions between the two.
	
	We say $C (\alpha, \gamma) A$ is a Galois connection if, for any choice of $c \in C$ and $a \in A$ we have
	\[
	\alpha(c) \preceq a \iff c \preceq \gamma(a)
	\]
\end{definition}

For our goals, we will call $C$ the \textit{concrete domain}, $A$ the \textit{abstract domain}, $\alpha$ the \textit{abstraction function} and $\gamma$ the \textit{concretization function}. The two functions $\alpha$ and $\gamma$ are also called \textit{adjoints}\footnote{Actually the term ``adjoint" comes from Category Theory, since Galois connections are adjunctions between the two posets seen as categories. However such a discussion would be outside the scope of this thesis, and would also require a basic knowledge of category theory.}.

\todo{Add meaning of the order on $C$ and $A$}
In program analysis we give the following intuitive meaning to those. $C$ is the domain of concrete states, $A$ the set of abstract properties we're interested in, $\alpha$ the function that maps a concrete state in the most precise abstract property that describes it and $\gamma$ a function that maps an abstract property in the biggest concrete state it describes.

\begin{example}[Intervals]\label{ch2:ex:intervals}
	We can now formalize the intuitive example \ref{intr:ex:intervals} of intervals we introduced in the previous chapter.
	$C$ is the set of possible values of the variable \code{x}. Since this is an integer values, elements of $C$ are subsets of $\setZ$, so $C = \pow(\setZ)$, with the ordering given by set inclusion. $A$ is the set of abstract properties we track in our analysis, so in our example the set of interval to which \code{x} may belongs. This means $A = \Int$.
	$\alpha$ is the function that allows us to abstract a set of possible values of \code{x} into the best (ie. most precise) possible abstract property:
	\begin{align*}
		\alpha(S) &= [\min(S); \max(S)]
	\end{align*}
	with the convention that $\min(\emptyset) = +\infty$, $\min(\emptyset) = -\infty$, the minimum of a lower-unbound set is $-\infty$ and the maximum of an upper-unbound set is $+\infty$. This abstraction function is exactly what we expect: no smaller interval can describe the set $S$, since $\min(S)$ and $\max(S)$ are elements of $S$ and so must also be in the interval. Conversely, this is a correct abstraction of $S$ since all its elements are between $\min(S)$ and $\max(S)$, so are in the interval too.
	
	$\gamma$ is the function that does the inverse operation: given an interval $[n, m]$, thought as a formal writing that describes the property that the value of \code{x} is between $n$ and $m$, gives us its ``meaning", that is the largest subset of $\setZ$ that matches that property:
	\[
	\gamma([n, m]) = \{ x \in \setZ \svert n \le x \le m \}
	\]
	The set $\{ x \in \setZ \svert n \le x \le m \}$ is exactly what is commonly represented with $[n; m]$: $\gamma$ is simply translating the formal writing (or, in our context, an abstract property) to a semantic set of values.

	Of course this definition of $\gamma$ is incomplete, the full, formal one being
	\begin{align*}
		\gamma([n, m]) &= \{ x \in \setZ \svert n \le x \le m \} \\
		\gamma([-\infty, m]) &= \{ x \in \setZ \svert x \le m \} \\
		\gamma([n, +\infty]) &= \{ x \in \setZ \svert n \le x \} \\
		\gamma([-\infty, +\infty]) &= \setZ \\
		\gamma([+\infty, -\infty]) &= \emptyset
	\end{align*}

	Showing that $\pow(\setZ) (\alpha, \gamma) \Int$ is a Galois connection is just a straightforward check. Fixed $S \in \pow(\setZ)$ and the interval $[n, m] \in \Int$ (for simplicity, we assume both $n$ and $m$ finite) we have
	\begin{align*}
		& \alpha(S) \preceq [n, m] \\
		\iff &[\min(S); \max(S)] \preceq [n, m] \\
		\iff &n \le \min(S),\, \max(S) \le m \\
		\iff &\forall x \in S\ .\ n \le x,\, \forall x \in S\ .\ x \le m \\
		\iff &S \subseteq \{ x \in \setZ \svert n \le x \le m \} \\
		\iff &S \subseteq \gamma([n, m])
	\end{align*}
\end{example}

We recall here two properties of Galois connection that will be useful in this thesis.

\begin{prop}\label{ch2:th:gc-extensive-charact}
	Let $C$ and $A$ be two partially ordered sets, and $\alpha : C \rightarrow A$, $\gamma : A \rightarrow C$ be a pair of monotone functions between the two.
	
	Then $C (\alpha, \gamma) A$ is a Galois connection if and only if both $\id_C \preceq \gamma \circ \alpha$ and $\alpha \circ \gamma \preceq \id_A$.
\end{prop}
\begin{prop}\label{ch2:th:gc-adjoints-preserve-glb-lub}
	Let $C (\alpha, \gamma) A$ be a Galois connection. Then $\gamma$ preserves greatest lower bounds and $\alpha$ preserves least upper bounds.
\end{prop}
In particular, this means that $\gamma$ maps $\top_A$ in $\top_C$ (because they are glb of the empty set) and dually $\alpha$ maps $\bot_C$ in $\bot_A$.

In a Galois connection it may very well happen that two abstract elements have the same concretization, as shown in the following example:
\begin{example}[Product abstraction]
	Consider two concrete domains, for instance two copies of $\pow(\setZ)$ that describes the possible states of two different variables \code{x} and \code{y} that appear in the program.
	Then consider the Galois connection $\pow(\setZ) (\alpha, \gamma) \Int$, and suppose we want to use it to abstract both variables. If we abstract separately both the element of $\pow(\setZ_x)$ and that of $\pow(\setZ_y)$ (where subscripts indicates the variable the domain refers to), we get a pair of intervals, one for \code{x} and one for \code{y}, that is an element of $\Int_x \times \Int_y$. It's easy to check that this abstraction function defines a Galois connection between $\pow(\setZ_x \times \setZ_y)$ and $\Int_x \times \Int_y$.
	
	However, this abstraction has redundant elements: consider
	\begin{align*}
		(\bot, [n, m])
	\end{align*}
	\[ ([n, m], \bot) \]
	\[ (\bot, \bot) \]
	where $\bot = [+\infty, -\infty]$ describes the empty interval. All these elements are concretized in the concrete $\emptyset \in \pow(\setZ_x \times \setZ_y)$.
\end{example}

We would like not to have those since they are different elements of the abstract domain that describes the same property. In analogy with logic, this is the same kind of redundancy as the possibility to describe the empty set in the following three different ways:
\[
\{ (x, y) \svert x \in \emptyset, n \le y \le m \}
\]
\[
\{ (x, y) \svert n \le x \le m, y \in \emptyset \}
\]
\[
\{ (x, y) \svert x \in \emptyset, y \in \emptyset \}
\]

In order to avoid this issue, we require $\gamma$ to be injective, so that no two different abstract elements can be concretized into the same concrete element, ie. they describe the same property. This turns out to give rise to an interesting definition, that of Galois insertion.
\begin{definition}[Galois insertion]\label{ch2:def:gi}
	Let $C (\alpha, \gamma) A$ be a Galois connection. We say this is a \textit{Galois insertion} if $\gamma$ is injective.
\end{definition}

The one proposed above isn't the standard definition of Galois insertion: more commonly, it requires $\alpha \circ \gamma$ to be identity on the abstract domain $\id_A$. However, the two are equivalent, as shown by the following characterization of Galois insertions.
\begin{prop}\label{ch2:th:gi-charact}
	Let $C (\alpha, \gamma) A$ be a Galois connection. Then the following are equivalent:
	\begin{enumerate}[label={(\arabic*)}]
		\item $\alpha \circ \gamma = \id_A$
		\item $\alpha$ is surjective
		\item $\gamma$ is injective
	\end{enumerate}
\end{prop}

The interval domain presented above is an example of Galois insertion: since we required $n \le m$ in the interval $[n, m]$ no two intervals describe the same concrete set, and hence $\gamma$ is injective. However, as we've seen, composing two independent interval domains gives rise to a Galois connection that isn't an insertion. There are techniques to delete redundant elements of an abstract domain in order to make it a Galois insertion (\todo{add references}), but we're not interested in them for this thesis.

In a Galois insertion, since $\gamma$ is injective, we have a bijection between $A$ and $\gamma(A)$. By the definition of Galois insertion we have $\alpha \circ \gamma = \id_A$, hence $\alpha$ is the inverse of $\gamma$ when restricted to $\gamma(A)$. Since both functions are monotone this defines an isomorphism of poset between $A$ and $\gamma(A) \subseteq C$.

Using this isomorphism, whenever we consider a Galois insertion we identify $A$ and its image, so that $A$ becomes a subset of $C$ and $\gamma = \id_A$. In this case, by proposition \ref{ch2:th:gc-extensive-charact} we have $\id_C \preceq \gamma \circ \alpha = \alpha$, corresponding to the intuitive idea that $\alpha$ must abstract a set of states in something bigger in order to over-approximate it.

Do note that this identification simplifies notation, but introduces a pitfall: reasoning as above we may be tempted to say that also $\alpha = \alpha \circ \gamma \preceq \id_A$, always by proposition \ref{ch2:th:gc-extensive-charact}, so concluding that $\id \preceq \alpha \preceq \id$ and hence $\alpha = \id$. However here we're neglecting the fact that $\gamma = \id_A$, not the identity on the whole set $C$, and actually $\gamma$ is defined only on elements of $A \subseteq C$. This means that the above relation is indeed correct, but only for elements of $A$, as pointed out by the fact that $\alpha \preceq \id_A$ and not $\id_C$. This is a problem that arise in general with this notation: two functions that looks the same are actually different because of their domain, that isn't specified. We'll make sure to always clarify the domain whenever it's not uniquely determined by the context.

We recall here a  way to define a Galois insertion
\begin{prop}\label{ch2:th:gi-moore-family}
	Let $C$ be a complete lattice, and $\bar{A} \subseteq C$ a \textit{Moore family}, that is a subset such that
	\begin{itemize}
		\item $\sup(C) \in \bar{A}$
		\item for any subset $T \subseteq \bar{A}$ of the Moore family, $\inf(T) \in \bar{A}$
	\end{itemize}
	Let also
	\[
	\alpha(x) = \inf\{ a \in \bar{A} \svert x \preceq a \}
	\]
	Then $C(\alpha, \id_C) \bar{A}$ is a Galois insertion.
\end{prop}

\subsection{Under-approximation Galois connections}
%The definition of Galois connection is not symmetric, in a way that can be seen from two different points of view. On the one hand we can say it's asymmetric in the two domains: if we swap concrete and abstract domain what we get is no longer a Galois connection. On the other hand, the definition puts $\gamma$ above and $\alpha$ below: in fact the two are also called upper and lower adjoints. This asymmetry in the two adjoints favours one specific direction, that is over-approximations.
%
%We say that the two asymmetries are actually the same because if we \textit{both} swap the two domains \textit{and} change the upper and lower adjoint definition, we get again a Galois connection

The definition of Galois connection is not symmetric, in the sense that the definition puts $\gamma$ above and $\alpha$ below: in fact the two are also called upper and lower adjoints respectively. This asymmetry in the two adjoints favours one specific direction, that is over-approximations, and is not suited to describe under-approximations. For this reason we introduce the notion of
\begin{definition}[Under-approximation Galois connection]\label{ch2:def:under-gc}
	Let $C$ and $A$ be two partially ordered sets, and $\alpha : C \rightarrow A$, $\gamma : A \rightarrow C$ be a pair of monotone functions between the two.
	
	We say $C (\alpha, \gamma) A$ is an under-approximation Galois connection if, for any choice of $c \in C$ and $a \in A$ we have
	\[
	a \preceq \alpha(c) \iff \gamma(a) \preceq c
	\]
\end{definition}

This definition is the same as that of Galois connection (\ref{ch2:def:gc}) but the fact that here $\alpha$ is above and $\gamma$ below.
\begin{example}\label{ch2:ex:intervals-0}
	Consider the following example of under-approximation Galois insertion (we haven't given its definition yet, but we're confident the reader can anticipate it): the concrete domain is $\pow(\setZ)$, while the abstract domain is the set of all intervals (see example \ref{ch2:ex:intervals}) containing $0$, plus the empty interval:
	\[
	\Int_0 = \{ I \in \Int \svert 0 \in I \} \cup \{ \bot \}
	\]
	where we used $\bot$ to represent the empty interval $[+\infty, -\infty]$.
	$\gamma$ is the identity since this is a (under-approximation) Galois insertion, and $\alpha(S)$ is the greatest interval fully contained in $S$ that includes $0$. Formally,
	\begin{align*}
		\alpha(S) = \bigcup \{ I \in \Int_0 \svert I \subseteq S \}
	\end{align*}
	The result is in $\Int_0$: if it isn't empty, it does indeed contain $0$, since is the union of intervals in $\Int_0$ that contains $0$ themselves. Moreover is an interval because union of overlapping intervals is an interval too, and all those intervals intersect at $0$.

	To show this is an under-approximation Galois connection, fix a set $S \subseteq \setZ$ and an interval $[-m, n]$ with $n, m \ge 0$.
	Now the following chain of equivalences hold:
	\begin{align*}
		&[-m, n] \subseteq \alpha(S) \\
		\iff &[-m, n] \subseteq \bigcup \{ I \in \Int_0 \svert I \subseteq S \} \\
		\iff &[-m, 0] \subseteq S, [0, n] \subseteq S \\
		\iff &\gamma([-m, n]) = [-m, n] \subseteq S
	\end{align*}
	hence the one proposed is an under-approximation Galois insertion.

	For simplicity we neglected the case where the interval is $\bot$ and assumed both $n, m$ are non negative integers, but those cases are analogous to the one we presented.
\end{example}

\todo{other example, intervals with 0 or 1}

With this definition, we could easily prove results analogous to propositions \ref{ch2:th:gc-extensive-charact}, \ref{ch2:th:gc-adjoints-preserve-glb-lub} and give an analogous definition of under-approximation Galois insertion.
Instead of doing this explicitly, we just observe that an under-approximation Galois connection $C (\alpha, \gamma) A$ is just a Galois connection between opposite domain $C^{\op} (\alpha, \gamma) A^\op$ to get as corollaries of those propositions analogous results where we just reverse inequalities:
\begin{prop}\label{ch2:th:under-gc-extensive-charact}
	Let $C$ and $A$ be two partially ordered sets, and $\alpha : C \rightarrow A$, $\gamma : A \rightarrow C$ be a pair of monotone functions between the two.
	
	Then $C (\alpha, \gamma) A$ is an under-approximation Galois connection if and only if both $\gamma \circ \alpha \preceq \id_C$ and $\id_A \preceq \alpha \circ \gamma$.
\end{prop}

\begin{prop}\label{ch2:th:under-gc-adjoints-preserve-lub-glb}
	Let $C (\alpha, \gamma) A$ be an under-approximation Galois connection. Then $\gamma$ preserves least upper bounds and $\alpha$ preserves greatest lower bounds.
\end{prop}

This second proposition has an interesting corollary for under-approximation Galois connection. Of course it's dual holds for ``normal" Galois connection, but in the under-approximation case is much more important:
\begin{corollary}\label{ch2:th:under-gc-union-closure}
	Let $\pow(C) (\alpha, \gamma) A$ be an under-approximation Galois connection. If $a, a' \in A$ then
	\[
	\gamma(a \sqcup a') = \gamma(a) \cup \gamma(a')
	\]
\end{corollary}

This corollary states that $A$ is closed under union: if we consider an under-approximation Galois connection, where $\gamma = \id_A$, then for any pair of sets $S, S' \subseteq C$ that are in $A$ (ie. they are abstract properties) we have that also $S \cup S' \in A$.
\todo{C'è un commento nel sorgente, è vero quello che dice?}%As remarked in the introduction, this property combined with difficulties in recovering from $\bot$ are the main reasons we believe under-approximation abstract domains are so hard to define.

We can also get an under-approximation version of proposition \ref{ch2:th:gi-moore-family} to define a Galois insertion from an opposite Moore family:
\begin{prop}\label{ch2:th:under-gi-moore-family}
	Let $C$ be a complete lattice, and $\bar{A} \subseteq C$ a subset such that
	\begin{itemize}
		\item $\inf(C) \in \bar{A}$
		\item for any subset $T \subseteq \bar{A}$ also $\sup(T) \in \bar{A}$
	\end{itemize}
	Let also
	\[
	\alpha(x) = \sup\{ a \in \bar{A} \svert a \preceq x \}
	\]
	Then $C(\alpha, \id_C) \bar{A}$ is an under-approximation Galois insertion.
\end{prop}

\section{Abstracting functions}
Definition of correct and best abstraction of a function $f: C \rightarrow C$. Examples.
Remark that with GI notation $\alpha \circ f = f^{\#} \circ \alpha$ but actually different domains.
Notation $f^{\flat}$ and examples.
Notion of complete abstraction.
\begin{definition}[Complete abstraction]\label{ch2:def:complete-abstr}
	Given a function $f$ and one of its correct abstractions $f^{\#}$, we say that it is a \textit{complete abstraction} when
	\[
	\alpha \circ f = f^{\#} \circ \alpha
	\]
\end{definition}
