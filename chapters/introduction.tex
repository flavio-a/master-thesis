\chapter{Introduction}

Thesis introduction.

\section{Abstract interpretation}
Static program analyses are a useful set of techniques to infer properties of programs from their source code, without executing them. Among those, abstract interpretation (\cite{cousot-77}\cite{cousot-79}) is a general framework to design analyses sound-by-construction.

Introduced for both over- and under-approximations, but used only for the former.\cite{cousot-77}\cite{cousot-79}

\begin{example}[Intervals]\label{intr:ex:intervals}
	\todo{Intuitive description of interval based analysis}
\end{example}

\section{Under-approximations}
Recent interest in under-approximations (see eg. \cite{ohearn-incorrectness-logic}), we would like to investigate whether this can be done in the setting of abstract interpretation.

\section{Contributions}
Brief description of the thesis subjects, highlighting new contributions/approaches. Chapters overview.

The goal of this thesis is to investigate the reasons that prevented the development of an under-approximation theory within the abstract interpretation framework.

We think one of the reasons this haven't happened is that "recover" from bottom is way harder than from top.