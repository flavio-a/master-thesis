\chapter{Conclusions and future works}

Some authors proposed \todo{ref} to use the union closure of well known over-approximation domains (eg. intervals) represented as sets of abstract elements in order to increase the precision of over-approximations. However this introduces overheads, hence the need for heuristics to decide when and how to reduce the size of this union. For instance, in the interval domain this means to decide when there are too many disjoint intervals, and then which ones should be merged into one, losing precision but gaining speed.
This approach also defines correct under-approximation domains, but again it needs heuristics to decide how to reduce the union size. In under-approximation, a safe possibility is to simply drop one interval from the union: the result is still an under-approximation of the real set of values. This idea correspond to application of the consequence rule of O'Hearn's incorrectness logic to drop a disjunction in the formula describing the final set of states, and is a perfect application of his sentence "For incorrectness reasoning, you must remember information as you go along a path, but you get to forget some of the paths."
The problem here is how to decide which intervals to drop, but a possibility would be to take inspiration from incorrectness logic and under-approximation tools not based on abstract interpretation. The question at that point would be whether there is any difference in using abstract interpretation or logic based tools, but it may be worth investigating.
